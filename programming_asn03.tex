%% Assignment #2 for Dynamic Programming and Greedy Algorithms

\documentclass[10pt,fullpage]{article}

\usepackage{amsmath,amssymb,amsthm,amsfonts} % Typical maths resource packages

\usepackage{graphics}                 % Packages to allow inclusion of graphics
\usepackage{graphicx}

\usepackage{hyperref}                 % For creating hyperlinks in cross references

\usepackage{listings}

\topmargin -1.5cm \oddsidemargin -0.04cm \evensidemargin -0.04cm
\textwidth 16.00cm \textheight 23.50cm
\parskip 7.2pt
\parindent 0.25in

\makeindex

\title{ Advanced Algorithms Assignment II }


\author{Matthew Bennett \\
{\small\em Dynamic Programming and Greedy Algorithms HW \  Draft
date \today }}

 \date{ }

\begin{document}
\maketitle

Bitonic Euclidean Traveling Salesperson Problem Due: February 7

This problem is taken from problem 16-1 from the textbook.

Professor Bitmeister has retired from the University and taken a
job selling bit buckets to major computer manufacturers. This
involves substantial traveling and Dr. Bitmeister wishes to reduce
the mileage on his Hase Kettwiesel.

You being a loyal graduate student of the old geezer have been
asked to provide a computer algorithm for Dr. Bitmeister to use to
plan his tour. He will provide you a sequence of geographic
coordinates for the cities he needs to visit. These will be
provided as a sequence of x and y values, one per line. He refuses
to tell you how many cities he will visit. You need to prepare for
an arbitrary number.

You should assume that Dr. Bitmeister will cycle from city to city
using the straightest possible path based on Euclidean distance
(Pythagorean Theorem!). You are to devise an optimal bitonic tour,
since arbitrary TSP tours could be too difficult to compute. A
bitonic tour will start with the left-most city and proceed
generally rightwards to the right-most city and then return to the
left-most city on a generally leftwards path. So it seems like
sorting on the x coordinates would be handy. For convenience, Dr.
Bitmeister has offered to accept a solution where the x
coordinates are all different.

Your output should a sequence of x and y values. Start with the
leftmost city and output 1 line per city on the tour. Repeat the
leftmost city's coordinates at the end.

You can create sample data sets using the script on orca:
~seyfarth/scripts/gen_cities. Supply a number of the command line
to indicate how many cities you want.

You can view your results using xmgrace. If you file is named
"results", simply enter "xmgrace results".
\end{document}
