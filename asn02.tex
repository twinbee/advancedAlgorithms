%% Assignment #2 for Dynamic Programming and Greedy Algorithms

\documentclass[10pt,fullpage]{article}

\usepackage{amsmath,amssymb,amsthm,amsfonts} % Typical maths resource packages

\usepackage{graphics}                 % Packages to allow inclusion of graphics
\usepackage{graphicx}

\usepackage{hyperref}                 % For creating hyperlinks in cross references

\usepackage{listings}

\topmargin -1.5cm \oddsidemargin -0.04cm \evensidemargin -0.04cm
\textwidth 16.00cm \textheight 23.50cm
\parskip 7.2pt
\parindent 0.25in

\makeindex

\title{ Advanced Algorithms Assignment II }


\author{Matthew Bennett \\
{\small\em Dynamic Programming and Greedy Algorithms HW \  Draft
date \today }}

 \date{ }

\begin{document}
\maketitle

\textbf{Exercise 15.2-1 Optimal Matrix Composite Product for
$<5,10,3,12,5,50,6>$}\\

Using a computer to get the M values, we have:\\

\begin{tabular}{|c|c|c|c|c|c|c|}
  \hline
   j/k & A & B & C & D & E & F \\
  \hline
   A & 0 & 150 & 330  & 405  & 1655  &  2010 \\
  \hline
   B &   & 0 & 360 & 330  &  2430 & 1950  \\
  \hline
   C &   &   & 0 & 180  &  930 & 1770  \\
  \hline
   D &   &   &   & 0 & 3000  & 1860  \\
  \hline
   E &   &   &   &   & 0 & 1500  \\
  \hline
   F &   &   &   &   &   & 0 \\
  \hline
\end{tabular}

The corresponding s values are:\\

\begin{tabular}{|c|c|c|c|c|c|c|}
  \hline
   j/k & A & B & C & D & E & F \\
  \hline
   A & 0 & 1 & 2  & 2  & 4  &  5 \\
  \hline
   B &   & 0 & 2 & 2  &  2 & 2  \\
  \hline
   C &   &   & 0 & 3  &  4 & 4  \\
  \hline
   D &   &   &   & 0 & 4  & 4  \\
  \hline
   E &   &   &   &   & 0 & 5  \\
  \hline
   F &   &   &   &   &   & 0 \\
  \hline
\end{tabular}

One optimal parenthetization is: $((A_1A_2)((A_3A_4)(A_5A_6)))$

\newpage

\lstset{language=c++} \lstset{linewidth=90mm}
\begin{lstlisting}
#include <iostream>

using namespace std;

 const int numMatr = 7;
 const int p[numMatr] = {5,10,3,12,5,50,6};
 const int n = numMatr - 1;

void PRINT_OPTIMAL_PARENS(int s[n][n], int i, int j);

int main() {
 int i, j, k, l, q;
 int m[n][n];
 int s[n][n];

 for (i = 1; i <= n; i++)
 {
  m[i][i] = 0;
 }
 for (l = 2; l <= n; l++)      //l is the chain length.
 {
   for (i = 1; i <= n - l + 1; i++)
   {
    j = i + l - 1;
    m[i][j] = 999999999;
    for (k = i; k <= j - 1; k++)
    {
     q = m[i][k] + m[k + 1][j] + p[i-1]*p[k]*p[j];
     if (q < m[i][j])
     {
           m[i][j] = q;
           s[i][j] = k;
     }
    }
  }
 }

 PRINT_OPTIMAL_PARENS(s, 1, n);

 system("pause");
}



void PRINT_OPTIMAL_PARENS(int s[n][n], int i, int j) {
 if (i == j) cout << "A_" << i;
 else
 {
  cout << "(";
  PRINT_OPTIMAL_PARENS(s, i, s[i][j]);
  PRINT_OPTIMAL_PARENS(s, s[i][j] + 1, j);
  cout << ")";
 }
}
        \end{lstlisting}

\newpage

\textbf{Exercise 15.4-1 Determine an LCS of $<1, 0, 0, 1, 0, 1, 0,
1>$ and $<0, 1, 0, 1, 1, 0, 1, 1, 0>$.}

10111110 from:\\
11\textbf{10111110} and \\
\textbf{101111}0\textbf{1}0\textbf{0}\\

\textbf{Exercise 15.4-3 Give a memoized version of LCS-LENGTH that
runs in $O(m \times n)$ time.}
\begin{lstlisting}
 int LCS_LENGTH(int X[m], int Y[n], int m, int n)
 {
  //c is the global memoization table, X and Y are strings

  if (c[m][n]) > -1 return c[m][n];
  if (m == 0 || n == 0) //empty row or col array
     c[m][n] = 0
     else
     {
      if (X[m] = Y[m]) c[m][n] = LCS_LENGTH(X, Y, m - 1, n - 1) + 1;
      else c[m][n] = max(
        LCS_LENGTH(X, Y, m, n - 1),  // whichever is
        LCS_LENGTH(X, Y, m - 1, n)   //  smaller
       );
     }
  return c[m][n];
 }
\end{lstlisting}

\newpage

\textbf{Exercises 16.2-4 Professor Midas drives an automobile from
Newark to Reno along Interstate 80. His car's gas tank, when full,
holds enough gas to travel n miles, and his map gives the
distances between gas stations on his route. The professor wishes
to make as few gas stops as possible along the way. Give an
efficient method by which Professor Midas can determine at which
gas stations he should stop, and prove that your strategy yields
an optimal solution. }\\

If the professor stays on Interstate 80 without deviating, his
only choice will be in how far down the road he chooses to refill.
The optimal strategy to minimize the number of stops is to go as
far as possible on each tank of petrol without refilling.\\

\textbf{Show that the optimal substructure property holds} Suppose
there are n places for Midas to refuel. Consider an optimal
solution of s stops, the one being at k. The rest of the optimal
solution must be also be an optimal solution to the subproblem of
the remaining n - k stations. How can we be sure that this is
true? Assume a better subproblem solution exists. It must
therefore have fewer than s - 1 stops, we could use it to come up
with a solution with fewer than s stops for the full problem,
contradicting our original claim. So the optimal substructure
property holds.\\

\textbf{Show that the greedy choice property yields an optimal
solution} First we will assume that the method outlined above is
the "greedy choice", as it is trivial to show. Now, assume that
there are stations along the way, and Midas makes the Greedy
choice at a given time by choosing to visit station k. If Midas
brother, Jonas chooses station k-1 at that step (same initial
conditions), then he could have chosen station k instead, which
means he made a suboptimal choice. So Midas' Greedy Strategy is an
optimal strategy.

\end{document}
